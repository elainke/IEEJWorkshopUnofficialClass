\documentclass[platex,fleqn]{ieej-tec2}% documentclassの宣言で処理エンジンを指定する。fleqnは数式左寄せのため
\usepackage{amsmath,amssymb,bm}% ams-LaTeXの色々を使う
\usepackage[dvipdfmx]{graphicx,color}% 
\usepackage{nidanfloat}% 2段ぶち抜きの図を下に入れる,2段ぶち抜きの図と1段以内の図が同頁に共存するために必要
%
% 論文番号(1頁目右上にあるやつ)
\論文番号{xx-xx-xx}
%
% 日本語タイトル
\jtitle{電気学会研究会\LaTeX{}スタイルファイルに関する研究}
%
% 英語タイトル
\etitle{A study on the style file for IEEJ workshop.}
% 
% 著者リスト
\authorlist{%
 \authorentry*{三好 正太}{Shota Miyoshi}{1} %発表者は*をつける \authorentry{日本語名}{英語名}{後述の\affiliateとの関連付け番号}
 \authorentry{電子 太郎}{Taro Denshi}{2}
 \authorentry{電子 二郎}{Jiro Denshi}{2}
 \authorentry{電子 三郎}{Saburo Denshi}{3}
 \authorentry{電子 四郎}{Shiro Denshi}{3}
}
% 
% 所属 [authorentryとの関連付け番号]
\affiliate[1]
 {東京大学}
 {The University of Tokyo}
\affiliate[2]
 {日本電機大学}
 {Nihon Denki University}
\affiliate[3]
 {日本電力}
 {Nihon Electric Power Company}
%
% authormetryの番号により改行
\breakaffiliate{1}
\breakaffiliate{2}
%
% 和文キーワード
\begin{jkeyword}
和文キーワード,キーワード2,キーワード3,キーワード4,キーワード5,キーワード6,キーワード7,キーワード8,キーワード9
\end{jkeyword}
%
%英文キーワード。最後に%を入れないと最後のキーワードと)との間に空白の入るバグあり
\begin{ekeyword}
Keywords, keyword2, keyword3, keyword4, keyword5, keyword6, keyword7, keyword8, keyword9, keyword10, keyword11%
\end{ekeyword}
%
% 参照いろいろ(ieej-tec2の内容から外れると思うので直接書き。要望あればieej-tec2.clsに移します)
\def\tabref#1{\tablename\ref{#1}}
\def\figref#1{\figurename\ref{#1}}
\def\secref#1{\ref{#1}節}
\def\formref#1{\eqref{#1}式}
%
%%%%%%%%%%%%%%%% preambleここまで %%%%%%%%%%%%%%%%
%
\begin{document}
%
% abstractは\begin{document}と\maketitleの間に書く
\begin{abstract}
The ○○○○○○○○○○○○○○○○○○○○○○○○○○○○○○○○○○○○○○○○○○○○○
○○○○○○○○○○○○○○○○○○○○○○○○○○○○○○○○○○○○○○○○○○○○○○○
○○○○○○○○○○○○○○○○○○○○○○○○○○○○○○○○○○○○○○○○○○○○○○○○○.%
\end{abstract}
\maketitle

%
%%%%%%%%%%%%%%%% ここから本文 %%%%%%%%%%%%%%%%
%
\section{大見出}
本文□(MS明朝 $+$ Computer Modern)□□□□□□□
□□□□5□□□□\scalebox{0.5}[1]{1}\scalebox{0.5}[1]{0}□□□□\scalebox{0.5}[1]{1}\scalebox{0.5}[1]{5}□□□□\scalebox{0.5}[1]{2}\scalebox{0.5}[1]{0}□□□□\scalebox{0.5}[1]{2}\scalebox{0.5}[1]{5}□
□□□□□□□□□。
□□□□□□□,□□□□□□□□□□□□□□□□□□□□□□。
□□□□□□□□□□□□□,□□□□□□□□□□□□□□□□□□。

\section{大見出}

\subsection{小見出}
小見出_段落□□□□□□□□
□□□□□□□□□□□□□□□□□□□□□□□□□□□□□□
引用文献\cite{IEEJformat}□□引用文献\cite{bib2,bib3}□□引用文献\cite{bib4,bib5,bib6,bib7}□□□□□□□□□。

本文□□□□□□□□□□□□□□□□□□□□□□□
□□□□5□□□□\scalebox{0.5}[1]{1}\scalebox{0.5}[1]{0}□□□□\scalebox{0.5}[1]{1}\scalebox{0.5}[1]{5}□□□□\scalebox{0.5}[1]{2}\scalebox{0.5}[1]{0}□□□□\scalebox{0.5}[1]{2}\scalebox{0.5}[1]{5}□
□□□□□□□□□□□□□□□□□□□□□□□□□□□□□□□□□□□□□□□□□□□。
\begin{enumerate}
\item
□□□□□□□□□□□□□□□□□□□□□□□□□□□□□□□□□□□□。
\item
□□□□□□□□□□□□□□□□□□□□□□□□□□□□□□□□□□□□。
\end{enumerate}

\subsection{小見出_文字}
小見出_段落□□□□□□□□□□□□□□□□□□□□□□□□□□□□□□□□□□□□□□
□□□□□□□□□□□□□□□□□□□□□□□□□□□□□□□□□□□□。

本文□□□□□□□□□□□□□□□□□□□□□□□
□□□□5□□□□\scalebox{0.5}[1]{1}\scalebox{0.5}[1]{0}□□□□\scalebox{0.5}[1]{1}\scalebox{0.5}[1]{5}□□□□\scalebox{0.5}[1]{2}\scalebox{0.5}[1]{0}□□□□\scalebox{0.5}[1]{2}\scalebox{0.5}[1]{5}□
□□□□□□□□□□□□□□□□□□□□□□□□□□□□□□□□□□□□□□□□□□□。

本文□□□□□□□□□□□□□□□□□□□□□□□□□□□□□□□□□□□□□□□□□□□
□□□□□□□□□□□□□□□□□□□□□□□□□□□□□□。
%
\begin{equation}
Z = X + Y
\end{equation}
%
□□本文(字下無)□□□□□□□□□□□□□□□□□□□□□□□□□。
%
\begin{align}
Z &= X + Y \notag\\
& \qquad Z = X + Y \text{\scriptsize {数式行(続)}} \\
\intertext{式説明□□□□□□\scalebox{0.5}[1]{1}\scalebox{0.5}[1]{0}□□□□\scalebox{0.5}[1]{1}\scalebox{0.5}[1]{5}□□□□\scalebox{0.5}[1]{2}\scalebox{0.5}[1]{0}□□□□\scalebox{0.5}[1]{2}\scalebox{0.5}[1]{5}□
□□□□□□□□□□□□□□□□□□□□□。
(ここはちゃんと実装していないので準拠でない)%
}
Z &= X + Y \times \frac AB
\end{align}

本文□□□□□□□□□□□□□□□□□□□□□□□
□□□□5□□□□\scalebox{0.5}[1]{1}\scalebox{0.5}[1]{0}□□□□\scalebox{0.5}[1]{1}\scalebox{0.5}[1]{5}□□□□\scalebox{0.5}[1]{2}\scalebox{0.5}[1]{0}□□□□\scalebox{0.5}[1]{2}\scalebox{0.5}[1]{5}□
□□□□□□□(\tabref{tab:example}参照)。

\begin{table}[b]
\centering
\caption{Title.}
\label{tab:example}
\tabcolsep=1.5truemm
\begin{tabular}{|c|c|c|c|c|}\hline
表内文字 & \hspace{4zw} &  \hspace{4zw} &  \hspace{4zw} &  \hspace{4zw} \\\hline
Text & & & & \\\hline
& & & & \\\hline
\end{tabular}
\par
\begin{minipage}{68truemm}
\scriptsize%
図表説明(左)□□\scalebox{0.5}[1]{1}\scalebox{0.5}[1]{0}□□□□\scalebox{0.5}[1]{1}\scalebox{0.5}[1]{5}□□□□\scalebox{0.5}[1]{2}\scalebox{0.5}[1]{0}□□□□\scalebox{0.5}[1]{2}\scalebox{0.5}[1]{5}□□□□□表の左右は,75mm以内□□□□□□□□□□□。
\end{minipage}
\end{table}

本文□□□□□□□□□□□□□□□□□□□□□□□
□□□□5□□□□\scalebox{0.5}[1]{1}\scalebox{0.5}[1]{0}□□□□\scalebox{0.5}[1]{1}\scalebox{0.5}[1]{5}□□□□\scalebox{0.5}[1]{2}\scalebox{0.5}[1]{0}□□□□\scalebox{0.5}[1]{2}\scalebox{0.5}[1]{5}□□□□□□□□□□□□□□□□□□□□□□□□□□□□□□□□□□□□□□□□□□□□
□□□□□□□□□□□□□□□□□□□□□□□□□□□□□□□□□□□□□□□□□□□
□□□□□□□□□□□□□□□□□□□□□□□□□□□□□□□□□□□□□□□□□□□
□□□□□□□□□□□□□□□□□□□□□□□□□□□□□□。

本文□□□□□□□□□□□□□□□□□□□□□□□
□□□□5□□□□\scalebox{0.5}[1]{1}\scalebox{0.5}[1]{0}□□□□\scalebox{0.5}[1]{1}\scalebox{0.5}[1]{5}□□□□\scalebox{0.5}[1]{2}\scalebox{0.5}[1]{0}□□□□\scalebox{0.5}[1]{2}\scalebox{0.5}[1]{5}□
□□□□□□□(\tabref{tab:nidanfloat:example}参照)。

\begin{table*}[t]
\centering
\caption{Title.}
\label{tab:nidanfloat:example}
\tabcolsep=1.5truemm
\begin{tabular}{|c|c|c|c|c|c|c|c|c|c|c|}\hline
表内文字 & \hspace{4zw} &  \hspace{4zw} &  \hspace{4zw} &  \hspace{4zw} &  \hspace{4zw} &  \hspace{4zw} &  \hspace{4zw} &  \hspace{4zw} &  \hspace{4zw} &  \hspace{4zw} \\\hline
Text & & & & & & & & & & \\\hline
& & & & & & & & & & \\\hline
\end{tabular}
\par
\begin{minipage}{\hsize}
\scriptsize\centering%
図表説明(中央)□□□□□□表の左右は,165mm以内□□□□□□
\end{minipage}
\end{table*}

本文□□□□□□□□□□□□□□□□□□□□□□□
□□□□5□□□□\scalebox{0.5}[1]{1}\scalebox{0.5}[1]{0}□□□□\scalebox{0.5}[1]{1}\scalebox{0.5}[1]{5}□□□□\scalebox{0.5}[1]{2}\scalebox{0.5}[1]{0}□□□□\scalebox{0.5}[1]{2}\scalebox{0.5}[1]{5}□□□□□□□□□□□□□□□□□□□□□□□□□□□□□□□□□□□□□□□□□□□□
□□□□□□□□□□□□□□□□□□□□□□□□□□□□□□□□□□□□□□□□□□□□
□□□□□□□□□□□□□□□□□□□□□□□□□□□□□□□□□□□□□□□□□□□
□□□□□□□□□□□□□□□□□□□□□□□□□□□□□□。

本文□□□□□□□□□□□□□□□□□□□□□□□
□□□□5□□□□\scalebox{0.5}[1]{1}\scalebox{0.5}[1]{0}□□□□\scalebox{0.5}[1]{1}\scalebox{0.5}[1]{5}□□□□\scalebox{0.5}[1]{2}\scalebox{0.5}[1]{0}□□□□\scalebox{0.5}[1]{2}\scalebox{0.5}[1]{5}□□□□□□□□□□□□□□□□□□□□□□□□□□□□□□□□□□□□□□□□□□□□
□□□□□□□(\figref{fig:example}参照)。

%% 貼るためのeps画像をtexから作成 %%
\bgroup
\catcode37=11
\gdef\percentcharacter{%}
\egroup
\newwrite\fout%
\immediate\openout\fout=ieejtec2figure.eps%
\immediate\write\fout{\percentcharacter !PS-Adobe-3.0 EPSF-3.0}%
\immediate\write\fout{\percentcharacter\percentcharacter BoundingBox: 0 0 100 60}%
\immediate\write\fout{gsave}%
\immediate\write\fout{1 setlinewidth}%
\immediate\write\fout{gsave 0.75 setgray 0.25 setlinewidth}%
\immediate\write\fout{newpath 0 15 moveto 100 0 rlineto stroke}%
\immediate\write\fout{newpath 0 30 moveto 100 0 rlineto stroke}%
\immediate\write\fout{newpath 0 45 moveto 100 0 rlineto stroke}%
\immediate\write\fout{grestore}%
\immediate\write\fout{newpath 0 0 moveto 100 50 lineto stroke}%
\immediate\write\fout{newpath 100 0 moveto 0 0 lineto 0 60 lineto stroke}%
\immediate\write\fout{grestore}%
\immediate\write\fout{\percentcharacter\percentcharacter EOF}%
\immediate\closeout\fout%
%% ここまで %%

\begin{figure}[b]
\centering
\includegraphics[width=70truemm]{ieejtec2figure.eps}
\par
(a) Graph 1.
\caption{Title.}
\label{fig:example}
\end{figure}


本文□□□□□□□□□□□□□□□□□□□□□□□
□□□□5□□□□\scalebox{0.5}[1]{1}\scalebox{0.5}[1]{0}□□□□\scalebox{0.5}[1]{1}\scalebox{0.5}[1]{5}□□□□\scalebox{0.5}[1]{2}\scalebox{0.5}[1]{0}□□□□\scalebox{0.5}[1]{2}\scalebox{0.5}[1]{5}□□□□□□□□□□□□□□□□□□□□□□□□□□□□□□□□□□□□□□□□□□□□
□□□□□□□□□□□□□□□□□□□□□□□□□□□□□□□□□□□□□□□□□□□□
□□□□□□□□□□□□□□□□□□□□□□□□□□□□□□□□□□□□□□□□□□□□
□□□□□□□□□□□□□□□□□□□□□□□□□□□□□□□□□□□□□□□□□□□
□□□□□□□□□□□□□□□□□□□□□□□□□□□□□□□□□□□□□□□□□□□
□□□□□□□□□□□□□□□□□□□□□□□□□□□□□□。

本文□□□□□□□□□□□□□□□□□□□□□□□
□□□□5□□□□\scalebox{0.5}[1]{1}\scalebox{0.5}[1]{0}□□□□\scalebox{0.5}[1]{1}\scalebox{0.5}[1]{5}□□□□\scalebox{0.5}[1]{2}\scalebox{0.5}[1]{0}□□□□\scalebox{0.5}[1]{2}\scalebox{0.5}[1]{5}□□□□□□□□□□□□□□□□□□□□□□□□□□□□□□□□□□□□□□□□□□□□
□□□□□□□□□□□□□□□□□□□□□□□□□□□□□□□□□□□□□□□□□□□□
□□□□□□□□□□□□□□□□□□□□□□□□□□□□□□□□□□□□□□□□□□□□
□□□□□□□□□□□□□□□□□□□□□□□□□□□□□□□□□□□□□□□□□□□
□□□□□□□□□□□□□□□□□□□□□□□□□□□□□□□□□□□□□□□□□□□
□□□□□□□□□□□□□□□□□□□□□□□□□□□□□□。

本文□□□□□□□□□□□□□□□□□□□□□□□
□□□□5□□□□\scalebox{0.5}[1]{1}\scalebox{0.5}[1]{0}□□□□\scalebox{0.5}[1]{1}\scalebox{0.5}[1]{5}□□□□\scalebox{0.5}[1]{2}\scalebox{0.5}[1]{0}□□□□\scalebox{0.5}[1]{2}\scalebox{0.5}[1]{5}□□□□□□□□□□□□□□□□□□□□□□□□□□□□□□□□□□□□□□□□□□□□
□□□□□□□□□□□□□□□□□□□□□□□□□□□□□□□□□□□□□□□□□□□□
□□□□□□□□□□□□□□□□□□□□□□□□□□□□□□□□□□□□□□□□□□□□
□□□□□□□□□□□□□□□□□□□□□□□□□□□□□□□□□□□□□□□□□□□
□□□□□□□□□□□□□□□□□□□□□□□□□□□□□□□□□□□□□□□□□□□
□□□□□□□□□□□□□□□□□□□□□□□□□□□□□□。

本文□□□□□□□□□□□□□□□□□□□□□□□
□□□□5□□□□\scalebox{0.5}[1]{1}\scalebox{0.5}[1]{0}□□□□\scalebox{0.5}[1]{1}\scalebox{0.5}[1]{5}□□□□\scalebox{0.5}[1]{2}\scalebox{0.5}[1]{0}□□□□\scalebox{0.5}[1]{2}\scalebox{0.5}[1]{5}□□□□□□□□□□□□□□□□□□□□□□□□□□□□□□□□□□□□□□□□□□□□
□□□□□□□□□□□□□□□□□□□□□□□□□□□□□□□□□□□□□□□□□□□□
□□□□□□□□□□□□□□□□□□□□□□□□□□□□□□□□□□□□□□□□□□□□
□□□□□□□□□□□□□□□□□□□□□□□□□□□□□□□□□□□□□□□□□□□
□□□□□□□□□□□□□□□□□□□□□□□□□□□□□□□□□□□□□□□□□□□
□□□□□□□□□□□□□□□□□□□□□□□□□□□□□□。

本文□□□□□□□□□□□□□□□□□□□□□□□
□□□□5□□□□\scalebox{0.5}[1]{1}\scalebox{0.5}[1]{0}□□□□\scalebox{0.5}[1]{1}\scalebox{0.5}[1]{5}□□□□\scalebox{0.5}[1]{2}\scalebox{0.5}[1]{0}□□□□\scalebox{0.5}[1]{2}\scalebox{0.5}[1]{5}□□□□□□□□□□□□□□□□□□□□□□□□□□□□□□□□□□□□□□□□□□□□
□□□□□□□□□□□□□□□□□□□□□□□□□□□□□□□□□□□□□□□□□□□□
□□□□□□□□□□□□□□□□□□□□□□□□□□□□□□□□□□□□□□□□□□□□
□□□□□□□□□□□□□□□□□□□□□□□□□□□□□□□□□□□□□□□□□□□
□□□□□□□□□□□□□□□□□□□□□□□□□□□□□□□□□□□□□□□□□□□
□□□□□□□□□□□□□□□□□□□□□□□□□□□□□□。

本文□□□□□□□□□□□□□□□□□□□□□□□
□□□□5□□□□\scalebox{0.5}[1]{1}\scalebox{0.5}[1]{0}□□□□\scalebox{0.5}[1]{1}\scalebox{0.5}[1]{5}□□□□\scalebox{0.5}[1]{2}\scalebox{0.5}[1]{0}□□□□\scalebox{0.5}[1]{2}\scalebox{0.5}[1]{5}□□□□□□□□□□□□□□□□□□□□□□□□□□□□□□□□□□□□□□□□□□□□
□□□□□□□□□□□□□□□□□□□□□□□□□□□□□□□□□□□□□□□□□□□□
□□□□□□□□□□□□□□□□□□□□□□□□□□□□□□□□□□□□□□□□□□□□
□□□□□□□□□□□□□□□□□□□□□□□□□□□□□□□□□□□□□□□□□□□
□□□□□□□□□□□□□□□□□□□□□□□□□□□□□□□□□□□□□□□□□□□
□□□□□□□□□□□□□□□□□□□□□□□□□□□□□□。

本文□□□□□□□□□□□□□□□□□□□□□□□
□□□□5□□□□\scalebox{0.5}[1]{1}\scalebox{0.5}[1]{0}□□□□\scalebox{0.5}[1]{1}\scalebox{0.5}[1]{5}□□□□\scalebox{0.5}[1]{2}\scalebox{0.5}[1]{0}□□□□\scalebox{0.5}[1]{2}\scalebox{0.5}[1]{5}□□□□□□□□□□□□□□□□□□□□□□□□□□□□□□□□□□□□□□□□□□□□
□□□□□□□□□□□□□□□□□□□□□□□□□□□□□□□□□□□□□□□□□□□□
□□□□□□□□□□□□□□□□□□□□□□□□□□□□□□□□□□□□□□□□□□□□
□□□□□□□□□□□□□□□□□□□□□□□□□□□□□□□□□□□□□□□□□□□
□□□□□□□□□□□□□□□□□□□□□□□□□□□□□□□□□□□□□□□□□□□
□□□□□□□□□□□□□□□□□□□□□□□□□□□□□□。

本文□□□□□□□□□□□□□□□□□□□□□□□
□□□□5□□□□\scalebox{0.5}[1]{1}\scalebox{0.5}[1]{0}□□□□\scalebox{0.5}[1]{1}\scalebox{0.5}[1]{5}□□□□\scalebox{0.5}[1]{2}\scalebox{0.5}[1]{0}□□□□\scalebox{0.5}[1]{2}\scalebox{0.5}[1]{5}□□□□□□□□□□□□□□□□□□□□□□□□□□□□□□□□□□□□□□□□□□□□
□□□□□□□□□□□□□□□□□□□□□□□□□□□□□□□□□□□□□□□□□□□□
□□□□□□□□□□□□□□□□□□□□□□□□□□□□□□□□□□□□□□□□□□□□
□□□□□□□□□□□□□□□□□□□□□□□□□□□□□□。

本文□□□□□□□□□□□□□□□□□□□□□□□
□□□□5□□□□\scalebox{0.5}[1]{1}\scalebox{0.5}[1]{0}□□□□\scalebox{0.5}[1]{1}\scalebox{0.5}[1]{5}□□□□\scalebox{0.5}[1]{2}\scalebox{0.5}[1]{0}□□□□\scalebox{0.5}[1]{2}\scalebox{0.5}[1]{5}□□□□□□□□□□□□□□□□□□□□□□□□□□□□□□□□□□□□□□□□□□□□
□□□□□□□□□□□□□□□□□□□□□□□□□□□□□□□□□□□□□□□□□□□□
□□□□□□□□□□□□□□□□□□□□□□□□□□□□□□□□□□□□□□□□□□□□
□□□□□□□□□□□□□□□□□□□□□□□□□□□□□□□□□□□□□□□□□□□
□□□□□□□□□□□□□□□□□□□□□□□□□□□□□□□□□□□□□□□□□□□
□□□□□□□□□□□□□□□□□□□□□□□□□□□□□□□□□□□□□□□□□□□
□□□□□□□□□□□□□□□□□□□□□□□□□□□□□□□□□□□□□□□□□□□
□□□□□□□□□□□□□□□□□□□□□□□□□□□□□□□□□□□□□□□□□□□
□□□□□□□□□□□□□□□□□□□□□□□□□□□□□□□□□□□□□□□□□□□
□□□□□□□□□□□□□□□□□□□□□□□□□□□□□□。

本文□□□□□□□□□□□□□□□□□□□□□□□
□□□□5□□□□\scalebox{0.5}[1]{1}\scalebox{0.5}[1]{0}□□□□\scalebox{0.5}[1]{1}\scalebox{0.5}[1]{5}□□□□\scalebox{0.5}[1]{2}\scalebox{0.5}[1]{0}□□□□\scalebox{0.5}[1]{2}\scalebox{0.5}[1]{5}□□□□□□□□□□□□□□□□□□□□□□□□□□□□□□□□□□□□□□□□□□□□
□□□□□□□□□□□□□□□□□□□□□□□□□□□□□□□□□□□□□□□□□□□
□□□□□□□□□□□□□□□□□□□□□□□□□□□□□□。

\begin{thebibliography}{99}
\bibitem{IEEJformat}
原稿の書き方$\mid$一般社団法人 電気学会, \\
\verb|http://www.iee.jp/?page_id=4843| (2018年7月20日閲覧)
\bibitem{bib2}
Name : “Title English”, 雑誌名, Vol.巻数, No.号数 p.000 (発行年)\\
著書名:「タイトル」,雑誌名,Vol.巻数,No.号数 p.ページ数 (発行年)
\bibitem{bib3}
Name and Name : “Title Eng”, 雑誌名, Vol.巻数, No.号数 pp.000-000 (発行年)\\
著書名・著書名:「タイトル」,雑誌名,Vol.巻数,No.号数 pp.ページ数 (発行年)
著書名:「タイトル」,雑誌名,Vol.巻数,No.号数 pp.ページ数 (発行年)
\bibitem{bib4}
著書名:「タイトル」,雑誌名,Vol.巻数,No.号数 pp.ページ数 (発行年)
\bibitem{bib5}
著書名:「タイトル」,雑誌名,Vol.巻数,No.号数 pp.ページ数 (発行年)
\bibitem{bib6}
著書名:「タイトル」,雑誌名,Vol.巻数,No.号数 pp.ページ数 (発行年)
\bibitem{bib7}
著書名:「タイトル」,雑誌名,Vol.巻数,No.号数 pp.ページ数 (発行年)
\bibitem{bib8}
著書名:「タイトル」,雑誌名,Vol.巻数,No.号数 pp.ページ数 (発行年)
\bibitem{bib9}
著書名:「タイトル」,雑誌名,Vol.巻数,No.号数 pp.ページ数 (発行年)
\bibitem{bib10}
著書名:「タイトル」,雑誌名,Vol.巻数,No.号数 pp.ページ数 (発行年)
\bibitem{bib11}
著書名:「タイトル」,雑誌名,Vol.巻数,No.号数 pp.ページ数 (発行年)
\end{thebibliography}


\end{document}